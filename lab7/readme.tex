%package list
\documentclass{article}
\usepackage[top=3cm, bottom=3cm, outer=3cm, inner=3cm]{geometry}
\usepackage{multicol}
\usepackage{graphicx}
\usepackage{url}
%\usepackage{cite}
\usepackage{hyperref}
\usepackage{array}
%\usepackage{multicol}
\newcolumntype{x}[1]{>{\centering\arraybackslash\hspace{0pt}}p{#1}}
\usepackage{natbib}
\usepackage{pdfpages}
\usepackage{multirow}
\usepackage[utf8]{inputenc}
\usepackage[normalem]{ulem}
\useunder{\uline}{\ul}{}
\usepackage{svg}
\usepackage{xcolor}
\usepackage{listings}
\lstdefinestyle{ascii-tree}{
    literate={├}{|}1 {─}{--}1 {└}{+}1 
  }
\lstset{basicstyle=\ttfamily,
  showstringspaces=false,
  commentstyle=\color{red},
  keywordstyle=\color{blue}
}
%\usepackage{booktabs}
\usepackage{caption}
\usepackage{subcaption}
\usepackage{float}
\usepackage{array}

\newcolumntype{M}[1]{>{\centering\arraybackslash}m{#1}}
\newcolumntype{N}{@{}m{0pt}@{}}


%%%%%%%%%%%%%%%%%%%%%%%%%%%%%%%%%%%%%%%%%%%%%%%%%%%%%%%%%%%%%%%%%%%%%%%%%%%%
%CREACIÓN DE VARIABLE
\newcommand{\itemEmail}{jcondoripin@unsa.edu.pe}
\newcommand{\itemStudent}{Juan José Condori Pinto}
\newcommand{\itemCourse}{Programación Web 2}
\newcommand{\itemCourseCode}{1702122}
\newcommand{\itemSemester}{III}
\newcommand{\itemUniversity}{Universidad Nacional de San Agustín de Arequipa}
\newcommand{\itemFaculty}{Facultad de Ingeniería de Producción y Servicios}
\newcommand{\itemDepartment}{Departamento Académico de Ingeniería de Sistemas e Informática}
\newcommand{\itemSchool}{Escuela Profesional de Ingeniería de Sistemas}


%AQUIIII: CAMBIA LA INFO DEL LAB
\newcommand{\itemAcademic}{2023 - A}
\newcommand{\itemInput}{Del 23 Junio 2023}
\newcommand{\itemOutput}{Al 03 Junio 2023}
\newcommand{\itemPracticeNumber}{06}
\newcommand{\itemTheme}{Django - Usando una plantilla para ver Destinos Turísticos}
%%%%%%%%%%%%%%%%%%%%%%%%%%%%%%%%%%%%%%%%%%%%%%%%%%%%%%%%%%%%%%%%%%%%%%%%%%%%


%PARA EL PIE DE PÁGINA
\usepackage{fancyhdr}
\pagestyle{fancy}
\fancyhf{}
\setlength{\headheight}{30pt}
\renewcommand{\headrulewidth}{1pt}
\renewcommand{\footrulewidth}{1pt}
\fancyhead[L]{\raisebox{-0.2\height}{\includegraphics[width=3cm]{img/logo_episunsa.png}}}
\fancyhead[C]{\fontsize{7}{7}\selectfont	\itemUniversity \\ \itemFaculty \\ \itemDepartment \\ \itemSchool \\ \textbf{\itemCourse}}
\fancyhead[R]{\raisebox{-0.2\height}{\includegraphics[width=1.2cm]{img/logo_abet}}}
\fancyfoot[L]{Estudiante Juan José Condori Pinto}
\fancyfoot[C]{\itemCourse}
\fancyfoot[R]{Página \thepage}

% para el codigo fuente
\usepackage{listings}
\usepackage{color, colortbl}
\definecolor{dkgreen}{rgb}{0,0.6,0}
\definecolor{gray}{rgb}{0.5,0.5,0.5}
\definecolor{mauve}{rgb}{0.58,0,0.82}
\definecolor{codebackground}{rgb}{0.95, 0.95, 0.92}
\definecolor{tablebackground}{rgb}{0.8, 0, 0}

\lstset{frame=tb,
	language=bash,
	aboveskip=3mm,
	belowskip=3mm,
	showstringspaces=false,
	columns=flexible,
	basicstyle={\small\ttfamily},
	numbers=none,
	numberstyle=\tiny\color{gray},
	keywordstyle=\color{blue},
	commentstyle=\color{dkgreen},
	stringstyle=\color{mauve},
	breaklines=true,
	breakatwhitespace=true,
	tabsize=3,
	backgroundcolor= \color{codebackground},
}


\begin{document}
        %CARÁTULA
        \vspace*{10px}
    	
        \begin{center}	
            \fontsize{17}{17} \textbf{ Informe de Laboratorio \itemPracticeNumber}
        \end{center}
        \centerline{\textbf{\Large Tema: \itemTheme}}
        %\vspace*{0.5cm}	
    
        \begin{flushright}
    	\begin{tabular}{|M{2.5cm}|N|}
    		\hline 
    		\rowcolor{tablebackground}
    		\color{white} \textbf{Nota} \\
    		\hline 
    			\\[30pt]
    		\hline 			
    	\end{tabular}
        \end{flushright}	

	\begin{table}[H]
		\begin{tabular}{|x{4.7cm}|x{4.8cm}|x{4.8cm}|}
			\hline 
			\rowcolor{tablebackground}
			\color{white} \textbf{Estudiante} & \color{white}\textbf{Escuela}  & \color{white}\textbf{Asignatura}   \\
			\hline 
			{\itemStudent \par \itemEmail} & \itemSchool & {\itemCourse \par Semestre: \itemSemester \par Código: \itemCourseCode}     \\
			\hline 			
		\end{tabular}
	\end{table}		
	
	\begin{table}[H]
		\begin{tabular}{|x{4.7cm}|x{4.8cm}|x{4.8cm}|}
			\hline 
			\rowcolor{tablebackground}
			\color{white}\textbf{Laboratorio} & \color{white}\textbf{Tema}  & \color{white}\textbf{Duración}   \\
			\hline 
			\itemPracticeNumber & \itemTheme & 04 horas   \\
			\hline 
		\end{tabular}
	\end{table}
	
	\begin{table}[H]
		\begin{tabular}{|x{4.7cm}|x{4.8cm}|x{4.8cm}|}
			\hline 
			\rowcolor{tablebackground}
			\color{white}\textbf{Semestre académico} & \color{white}\textbf{Fecha de inicio}  & \color{white}\textbf{Fecha de entrega}   \\
			\hline 
			\itemAcademic & \itemInput &  \itemOutput  \\
			\hline 
		\end{tabular}
	\end{table}

\section{Temas a tratar}
        \begin{itemize}
            \item Proyectos de Django
            \item Aplicaciones en Django
            \item Relaciones uno a muchos en Django
            \item Relaciones de muchos a muchos en Django
            \item Django PDF y emails
        \end{itemize}
    
\section{Ejercicios}
\begin{itemize}
    \item Deberán replicar la actividad del video donde se obtiene una plantilla de una aplicación de Destinos turísticos y adecuarla a un proyecto en blanco Django.
    \item Luego trabajar con un modelo de tabla DestinosTuristicos donde se guarden nombreCiudad, descripcionCiudad, imagenCiudad, precioTour, ofertaTour (booleano). Estos destinos turísticos deberán ser agregados en una vista dinámica utilizando tags for e if.
    \item Para ello crear una carpeta dentro del proyecto github colaborativo con el docente, e informar el link donde se encuentra.
    \item Crear formularios de Añadir Destinos Turísticos, Modificar, Listar y Eliminar Destinos.
    \item Eres libre de agregar CSS para decorar tu trabajo.
    \item Ya sabes que el trabajo con Git es obligatorio. Revisa el avance de la teoría Django parte 4
\end{itemize} 

\section{\textcolor{red}{Rubricas}}
        \subsection{\textcolor{red}{Rubrica para entregable Informe}}
            \begin{table}[ht]
                \centering
                \caption{Rúbrica para tipo de informe}
                \begin{tabular}{
                        |p{2.5cm}
                        |p{6cm}
                        |p{2cm}
                        |p{2cm} |}
                    \hline
                        \multicolumn{2}{|c|}{\textbf{Informe}} & \centerline{\textbf{Cumple}} & \centerline{\textbf{No cumple}} \\
                    \hline
                        \textbf{\textcolor{red}{Latex}} & \textcolor{blue}{El informe esta en formato PDF desde Latex, con un formato limpio (buena presentación) y facil de leer.} & \centerline{20} & \centerline{0} \\
                    \hline
                        \textbf{MarkDown} & El informe esta en formato PDF desde MarkDown README.md, con un formato limpio (buena presentacion) y facil de leer. & \centerline{17} & \centerline{0}\\
                    \hline
                        \textbf{MS Word} & El informe esta en formato PDF desde plantilla MS Word, con un formato limpio (buena presentacion) y facil de leer. & \centerline{15} & \centerline{0}\\
                    \hline
                        \textbf{Observaciones} & Por cada observacion se le descontara puntos. & \centerline{-} & \centerline{-}\\
                    \hline
                    \end{tabular}
                \label{tab:tab1}
            \end{table}
        \subsection{\textcolor{red}{Rubrica para el contenido del Informe y demostracion}}
            \begin{itemize}
                \item El alumno debera marcar o dejar en blanco en las celdas de la columna Checklist, deacuerdo a si cumplio o no con el ́ıtem correspondiente.
                \item Si un alumno supera la fecha de entrega, su calificacion siempre sera sobre la nota mınima aprobada, siempre y cuando cumpla con todos lo items.
                \item El alumno debe autocalificarse en la columna Estudiante de acuerdo a la tabla de calificacion de niveles de desempeño:
            \end{itemize}
            \begin{table}[ht]
                \centering
                \caption{Niveles de desempeño}
                \begin{tabular}{
                        >{\centering\arraybackslash}m{1.2cm}
                        >{\centering\arraybackslash}m{3cm}
                        >{\centering\arraybackslash}m{3cm}
                        >{\centering\arraybackslash}m{3cm}
                        >{\centering\arraybackslash}m{3cm}}
                    \hline
                    \multicolumn{5}{c}{Nivel} \\
                    \hline
                    \textbf{Puntos} & Insatisfactorio 25\% & En Proceso 50\% & Satisfactorio 75\% & Sobresaliente 100\% \\
                    \textbf{2.0} & 0.5 & 1.0 & 1.5 & 2.0 \\
                    \textbf{4.0} & 1.0 & 2.0 & 3.0 & 4.0 \\
                    \hline
                \end{tabular}
                \label{tab:tab2}
            \end{table}
            \begin{table}[]
                \centering
                \caption{Rubrica para contenido del Informe y demostracion}
                \begin{tabular}{
                    |m{2.5cm}
                    |m{7cm}
                    |>{\centering\arraybackslash}m{1cm}
                    |>{\centering\arraybackslash}m{1.2cm}
                    |>{\centering\arraybackslash}m{1.5cm}
                    |>{\centering\arraybackslash}m{1.2cm}|}
                    
                    \hline
                    \multicolumn{2}{|c|}{Contenido y demostracion} & Puntos & Checklist & Estudiante & Profesor \\
                    \hline
                    \textbf{1. GitHub} & Hay enlace URL activo del directorio para el laboratorio hacia su repositorio GitHub con codigo fuente terminado y facil de revisar. & 2 & X & 2 &   \\
                    \hline
                    \textbf{2. Commits} & Hay capturas de pantalla de los commits mas importantes con sus explicaciones detalladas. (El profesor puede preguntar para refrendar calificacion). & 4 & X & 3 &   \\
                    \hline
                    \textbf{3. Código fuente} & Hay porciones de codigo fuente importantes con numeracion y explicaciones detalladas de sus funciones. & 2 & X & 2 &   \\
                    \hline
                    \textbf{4. Ejecucion} & Se incluyen ejecuciones/pruebas del codigo fuente explicadas gradualmente. & 2 & X & 2 &   \\
                    \hline
                    \textbf{5. Pregunta} & Se responde con completitud a la pregunta formulada en la tarea. (El profesor puede preguntar para refrendar calificacion). & 2 & X & 2 &   \\
                    \hline
                    \textbf{6. Fechas} & Las fechas de modificacion del codigo fuente estan dentro de los plazos de fecha de entrega establecidos. & 2 & X & 2 &   \\
                    \hline
                    \textbf{7. Ortografia} & El documento no muestra errores ortograficos. & 2 & X & 2 &   \\
                    \hline
                    \textbf{8. Madurez} & El Informe muestra de manera general una evolucion de la madurez del codigo fuente, explicaciones puntuales pero precisas y un acabado impecable. (El profesor puede preguntar para refrendar calificacion). & 4 & X & 3 &   \\
                    \hline
                    \multicolumn{2}{|c|}{Total} & 20 &  & 18 & \\
                    \hline
                \end{tabular}
                \label{tab:tab3}
            \end{table}
    
\section{Referencias}
        \begin{itemize}
            \item \url{https://developer.mozilla.org/en-US/docs/Learn/Server-side/Django/Tutorial_local_library_website}
            \item \url{https://github.com/mdn/django-locallibrary-tutorial}
            \item \url{https://github.com/rescobedoq/pw2/tree/main/labs/lab05}
            \item William S. Vincent. (2022). Django for Beginners: Build websites with Python. Django 4.0.leanpub.com [\href{http://library.lol/main/22AF742D96697DE55EF5F88B08F1AA86}{URL}]
            \item \url{https://docs.djangoproject.com/en/4.1/ref/models/fields/}
            \item \url{https://docs.djangoproject.com/en/4.0/topics/db/examples/many_to_many/}
            \item \url{https://docs.djangoproject.com/en/4.0/topics/db/examples/many_to_one/}
            \item \url{https://blog.hackajob.co/djangos-new-database-constraints/}
            \item \url{https://stackoverflow.com/questions/3330435/is-there-an-sqlite-equivalent-to-mysqls-describe-table}
            \item \url{https://docs.djangoproject.com/en/4.1/ref/validators/#how-validators-are-run}
            \item \url{https://docs.djangoproject.com/en/4.1/ref/models/instances/}
        \end{itemize}
\end{document}  